\documentclass[11pt]{article}
\usepackage{report}

\begin{document}
\section{Problem 1}
\subsection{Poisson}
For the poisson equation problem we have
\begin{align}
	-\nabla (a(x,y)\nabla u) &= f,\quad x\in \Omega \\
	u &= 0,\quad x\in \delta \Omega.
\end{align}
In order to rewrite this equation on weak, or variational, form we start with multiplying the integral of $f$ with a test function $v$ where $v\in V_0$ where $V_0 = \{ ||v||^2_{L^2} + ||\nabla v||^2 < \infty, v|_{\nabla \Omega} = 0\}$. This results in
\begin{align}
	\int_{\Omega} f v d s &= - \int_{\Omega} \nabla (a \nabla u) v ds \\
	&= \int_{\Omega} a \nabla u \nabla v ds - \int_{\delta \Omega} v \nabla u d\bar{l} \\
	&= \int_{\Omega} a \nabla u \nabla v ds
\end{align}
where we used greens identity in the first step and the fact that the test function is zero at the boundary in the second step.
\subsection{Shroedinger}
When we look at the time dependent solution we have a significantly different problem.
\begin{align}
	\nabla^2 u + a(x,y) u &= f, \quad x\in \Omega \\
	u &= 0, \quad x \in \delta \Omega
\end{align}
We continue and do the same thing as we did for the poisson problem. Starting with multiplying the integral of f with a test function $v \in V_0$.
\begin{align}
	\int_{\Omega} f v d s &= - \int_{\Omega} \nabla^2 u v ds + \int_{\Omega} auv ds \\
	&= \int_{\Omega} \nabla u\nabla vds-\int_{\delta\Omega} v \nabla u d\bar{l} + \int_{\Omega} auvds \\
	&= \int_{\Omega} \nabla u \nabla v ds + \int_{\Omega} a u v ds
\end{align}
here the exact same steps as for the poisson derivation was made. The difference of course is the extra term $\int_{\Omega} a u v ds$ which is not within the derivate sign in the initial problem setup.

\section{Problem 2}

	

\end{document}
