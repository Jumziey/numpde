\documentclass[11pt]{article}
\usepackage{report}

\begin{document}
\section{Problem 1}
\subsection{Poisson}
For the poisson equation problem we have
\begin{align}
	-\nabla (a(x,y)\nabla u) &= f,\quad x\in \Omega \\
	u &= 0,\quad x\in \delta \Omega.
\end{align}
In order to rewrite this equation on weak, or variational, form we start with multiplying the integral of $f$ with a test function $v$ where $v\in V_0$ where $V_0 = \{ ||v||^2_{L^2} + ||\nabla v||^2 < \infty, v|_{\nabla \Omega} = 0\}$. This results in
\begin{align}
	\int_{\Omega} f v d s &= - \int_{\Omega} \nabla (a \nabla u) v ds \\
	&= \int_{\Omega} a \nabla u \nabla v ds - \int_{\delta \Omega} a v \nabla u d\bar{l} \\
	&= \int_{\Omega} a \nabla u \nabla v ds
\end{align}
where we used greens identity in the first step and the fact that the test function is zero at the boundary in the second step.
\subsection{Shroedinger}
When we look at the time dependent solution we have a significantly different problem.
\begin{align}
	\nabla^2 u + a(x,y) u &= f, \quad x\in \Omega \\
	u &= 0, \quad x \in \delta \Omega
\end{align}
We continue and do the same thing as we did for the poisson problem. Starting with multiplying the integral of f with a test function $v \in V_0$.
\begin{align}
	\int_{\Omega} f v d s &= - \int_{\Omega} \nabla^2 u v ds + \int_{\Omega} auv ds \\
	&= \int_{\Omega} \nabla u\nabla vds-\int_{\delta\Omega} v \nabla u d\bar{l} + \int_{\Omega} auvds \\
	&= \int_{\Omega} \nabla u \nabla v ds + \int_{\Omega} a u v ds
\end{align}
here the exact same steps as for the poisson derivation was made. The difference of course is the extra term $\int_{\Omega} a u v ds$ which is not within the derivate sign in the initial problem setup.

\section{Problem 2}
\subsection{Poisson}
In this problem we add some more complex boundary conditions. The poisson equation becomes 
\begin{align}
	-\Delta(a(x,y)\Delta u) &= f, \quad x\in\Omega \\
	c_0 u + c_1 \nabla u \cdot \hat{n} &= 0, \quad x \in\delta\Omega
\end{align}
We still use the same kind process of deriving the weak or variational form. One slight difference is that if u is gonna exists with in the test function space we need to change it a bit, so that it can be none zero at the boundary. Thus now we multiply the integral of f with a test function $v\in V_1$ where $V_1 = \{||v||^2_{L^2} + ||v'||^2_{L^2} < \infty\}$.
\begin{align}
	\int_{\Omega} f v ds &= - \int_{\Omega} \nabla (a\nabla u) ds \label{eq:poisGenVar}\\
	&= \int_{\Omega} a \nabla u \nabla v ds - \int_{\delta \Omega}a v \nabla u \cdot \hat{n} dl \\
	& = \int_{\Omega} a \nabla u \nabla v ds + \frac{c_0}{c_1}  \int_{\delta \Omega} a v u dl
\end{align}

\subsubsection{Energy norm}
For the energy norm we simply look at the variational form and make sure the terms including $u$ is on one side and take that side and put $v=u$ in order to get the an expression for the energy norm $||u||^2_E$
\begin{equation}
	||u||^2_E = \int_{\Omega} a (\nabla u)^2 ds + \frac{c_0}{c_1}  \int_{\delta \Omega} a u^2 dl
\end{equation}
This physically represents the energy in the system, while the left hand side of equation \ref{eq:poisGenVar} stands for the load on the system.

\subsection{Shroedinger}
Here we apply the same more complex boundary condidion to the time-independent shroedinger equation. 
\begin{align}
	\nabla^2 u + a(x,y) u &= f, \quad x\in \Omega \\
	c_0 u + c_1 \nabla u \cdot \hat{n} &= 0, \quad x \in \delta \Omega
\end{align}
And in order to get to the variational form we multiply the integral of f with a test function $v\in V_1$. 
\begin{align}
	\int_{\Omega} f v d s &= - \int_{\Omega} \nabla^2 u v ds + \int_{\Omega} auv ds \\
	&= \int_{\Omega} \nabla u\nabla vds-\int_{\delta\Omega} v \nabla u \cdot \hat{n}dl + \int_{\Omega} auvds \\
	&= \int_{\Omega} \nabla u \nabla v ds + \int_{\Omega} a u v ds + \frac{c_0}{c_1}\int_{\delta \Omega} u v dl 
\end{align}
\subsubsection{Energy norm}
For the energy norm we, again, look at the variational form and make sure the terms including $u$ is on one side and take that side and put $v=u$ in order to get the an expression for the energy norm $||u||^2_E$
\begin{equation}
	||u||^2_E = \int_{\Omega} (\nabla u)^2 ds + \int_{\Omega} a u^2 ds + \frac{c_0}{c_1}\int_{\delta \Omega} u^2 dl 
\end{equation}

\section{Problem 3}
In this exercise we wanna use the previous results and show how a finite elements approximation might look like. We will use the piecewise linear approximation. This means we will use a subspace of $V_0$ and $V_1$ which can be spanned by the hat functions, as define in the course book, that we will write as $\{\phi\}^{n_i}_{i=1}$.
\subsection{Poisson}
\subsubsection{Problem 1 Formulation}
We start off with
\begin{equation}
	\int_{\Omega} f v d s = \int_{\Omega} a \nabla u \nabla v ds
\end{equation}
Rewritting $v$ in terms of hat functions and we approximate $u$ as $u_h$ which is basically a solution projected onto the space spanned by the hat functions we get
\begin{equation}
	\int_{\Omega} f \phi_i ds = \sum^n_i \xi_i (\int_{\Omega} a \nabla \phi_i \nabla \phi_j ds)
\end{equation}
Which is what we where seeking.

\subsubsection{Problem 2 Forumlation}
In this exercise we do the same thing as in the previous but we have to take into account the extra terms given by the boundary conditions. We start off with
\begin{equation}
	\int_{\Omega} f v ds = \int_{\Omega} a \nabla u \nabla v ds + \frac{c_0}{c_1}  \int_{\delta \Omega} a v u dl.
\end{equation}
Doing the same thing to this expression as in the previous exercise we end up with
\begin{align}
	\int_{\Omega} f \phi_i ds &=  \sum^n_i \xi_i (\int_{\Omega} a \nabla \phi_i \nabla \phi_j ds) + \sum^n_i \frac{c_0}{c_1} \xi_i (\int_{\delta \Omega} a \phi_i \phi_j ds) \\
	&= \sum^n_i \xi_i ( \int_{\Omega} a \nabla \phi_i \nabla \phi_j ds + \frac{c_0}{c_1} \int_{\delta \Omega} a \phi_i \phi_j ds).
\end{align}
Here one has to care and realize that the last term in the expression is only on the boundary of the two dimensional geometry. 
\subsection{Shroedinger}
\subsubsection{Problem 1 Formulation}
In this exercise we start off with the variational formulation
\begin{equation}
	\int_{\Omega} fv ds= \int_{\Omega} \nabla u \nabla v ds + \int_{\Omega} a u v ds
\end{equation}
Using the subspace of $V_0$ that is spanned by the hat functions for the test function and solution space we get
\begin{align}
	\int_{\Omega} f \phi_i ds &= \sum^n_i \xi_i \int_{\Omega} \nabla \phi_i \nabla \phi_j ds + \sum^n_i \xi_i \int_{\Omega}a \phi_i \phi_j ds \\
	&= \sum^n_i \xi_i ( \int_{\Omega} \nabla \phi_i \nabla \phi_j ds + \int_{\Omega}a \phi_i \phi_j ds).
\end{align}
Here we can clearly see the use of both a stiffness matrix and a mass matrix. 

\subsubsection{Problem 2 Formulation}
Here we start of the variational forumalation
\begin{equation}
	\int_{\Omega} f v ds = \int_{\Omega} \nabla u \nabla v ds + \int_{\Omega} a u v ds + \frac{c_0}{c_1}\int_{\delta \Omega} u v dl 
\end{equation} 
Using the subspace of $V_0$ that is spanned by the hat functions for the test function and solution space we get
\begin{align}
	\int_{\Omega} f \phi_i ds &= \sum^n_i \xi_i \int_{\Omega} \nabla \phi_i \nabla \phi_j ds + \sum^n_i \xi_i \int_{\Omega} a \phi_i \phi_j  ds + \frac{c_0}{c_1} \sum^n_i \xi_i \int_{\delta \Omega} \phi_i \phi_j ds \\
	&= \sum^n_i \xi_i ( \int_{\Omega} \nabla \phi_i \nabla \phi_j ds + \int_{\Omega} a \phi_i \phi_j  ds + \frac{c_0}{c_1} \int_{\delta \Omega} \phi_i \phi_j ds )
\end{align}
Here we note we once again only get some extra terms on the boundary due to the more general formulation of the boundary conditions. 

\section{Problem 4}






\end{document}

